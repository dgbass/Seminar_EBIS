%%
%%Fuer Korrektur eigene Environment fuer Kommentare,etc.
\newenvironment{altertext}{\begin{quote}\small}{\end{quote}}
\usepackage{tikz,ed,fixme} 
\usepackage[framemethod=tikz]{mdframed} \newcommand{\notab}[1]{\textbf{\sffamily\textcolor{blue}{ [[#1]] }}} \mdfdefinestyle{leftbar}{linecolor=blue,linewidth=3pt,topline=false,
rightline=false,bottomline=false}  \mdfdefinestyle{rahmen}{roundcorner=10pt,leftmargin=1, rightmargin=1, linecolor=blue,outerlinewidth=1, innerleftmargin=15, innertopmargin=15,innerbottommargin=15,  skipabove=\topskip,skipbelow=\topskip} \newenvironment{nota}{\begin{quote}\sffamily\hrule}   {\par\smallskip\hrule\end{quote}} 
\newenvironment{VB}{\begin{mdframed}[style=leftbar]\noindent\sffamily\parindent0pt}
{\end{mdframed}\noindent} \newenvironment{VB*}{\begin{quote}\color{blue}}{\end{quote}}  \newcommand{\Del}[2][]{\footnote{Entfernt: \textcolor{brown}{#2} --- #1}} 


%%
%&Anpassung Raender Dokument 
\usepackage{geometry}
%\geometry{a4paper, top=30mm, left=20mm, right=20mm, bottom=30mm,footskip=12mm}
\geometry{a4paper, top=30mm, bottom=30mm,footskip=12mm}

%%%Durchlaufende Nummerierung Fußnoten \footnote
\usepackage{remreset}
\makeatletter
\@removefromreset{footnote}{chapter}
\makeatother
%Fussnote in Table sichtbar
\usepackage{footnote}
\makesavenoteenv{tabular}
\makesavenoteenv{table}
%%%

%%
%% Unterstützung für deutsche Sprache, Umlaute etc.
%%
\usepackage{ngerman}
%\usepackage [ngerman] {babel}
\usepackage [utf8] {inputenc}
\usepackage{csquotes}
\usepackage[T1]{fontenc}
\usepackage{textcomp}

%%Erzeugt für jedes Chapter eine neue Start-Seite
%%mit Nummer des Kapitels in Grau
%\usepackage{longtable}					%js
%\usepackage[grey,helvetica]{quotchap}	%js

%%Zeigt nur die verwendeten Abkuerzungen im Abkuerzungsverzeichnis
\usepackage[printonlyused]{acronym}		%js


%%
%% Diverse Pakete
%%
\definecolor{dunkelrot}{rgb}{102,0,0}
\usepackage{graphicx} 
\usepackage{float}
\usepackage{verbatim}
\usepackage{fancyhdr}
\usepackage{floatflt} 
\usepackage{enumitem}
\usepackage{mathrsfs,amssymb}
\usepackage[intlimits] {empheq}
\usepackage{theorem}
\usepackage{wrapfig}
\usepackage {marvosym}
\usepackage {xcolor}
\usepackage{caption}
\usepackage{subcaption}
\usepackage{scrhack}
\usepackage{microtype}
\usepackage{setspace}
\usepackage{array, tabularx}
\usepackage{amsmath}
\usepackage{booktabs}
\usepackage{ragged2e}
\usepackage{multirow}
\usepackage[left]{eurosym}	
\usepackage[section]{placeins}			%js
\usepackage{epigraph}					%js
\setlength{\epigraphwidth}{0.4\textwidth}
\usepackage{framed}

%%
%% Anpassung an Aussehen bei Click auf Links und Verlinkungen des Inhaltsverzeichnis
\usepackage[hyphens]{url}

\usepackage[colorlinks,pdfpagelabels,pdfstartview = FitH,bookmarksopen = true,bookmarksnumbered = true,linkcolor = black,plainpages = false,hypertexnames = false,citecolor = black, urlcolor=black] {hyperref}

%%use the same style for url as in the document
\urlstyle{same}
%%

%%
% Anpassung der Ueberschriften
\usepackage{titlesec}
\titlespacing{\chapter}{0pt}{-6em}{6pt}
\titlespacing{\section}{0pt}{6pt}{6pt}
\usepackage{listingsutf8}

%%
% Anpassung Zeilenabstand
\linespread {1.5}


%%
%% Verhindert überhängende Absatzteile
%%
\clubpenalty10000
\widowpenalty10000
\displaywidowpenalty=10000

%%Anpassung schreiben auf 2. Seite nach chapter
\renewcommand*\chapterheadstartvskip{\vspace*{-0.5cm}}
%%
\setlength{\parindent}{0pt}
\setlength{\parskip}{1.4ex plus 0.35ex minus 0.3ex}

% Schrift ----------------------------------------------------------------------
\usepackage{lmodern} % bessere Fonts
%\usepackage{relsize} % Schriftgröße relativ festlegen
\usepackage{microtype} % AL: Verbessert den Schriftsatz

%%
%% Schriftarten
%%
\renewcommand{\sfdefault}{phv}
\renewcommand{\ttdefault}{ppl}
\renewcommand{\rmdefault}{phv}
%\KOMAoptions{DIV=last}
% change Font during text
\newcommand{\changefont}[3]{
\fontfamily{#1}\fontseries{#2}\fontshape{#3}\selectfont}


%%
%% Formatierung des Literaturverzeichnisses
%%
%\bibliographystyle{plaindin} % Nummern und alphabetisch sortiert
\bibliographystyle{alphadin} % Buchstaben und sortiert
%\bibliographystyle{abbrvdin} % Nummern und abgekürzte Namen
%\bibliographystyle{unsrtdin} % Nummern und unsortiert

% fortlaufendes Durchnummerieren der Fussnoten ----------------------------------
\usepackage{chngcntr}

%%
%% Trennungsregeln
%%
\hyphenation{Sil-ben-trenn-ung}

%%
%% Seitenlayout
%%
\pagestyle{fancy}

\setcounter{tocdepth}{3}	% tiefe des inhaltsverzeichnisses
\setcounter{secnumdepth}{3} % Nummerierung der Überschriftentiefe festlegen


%%
%% Schönere Bullets bei Aufzählungen
%%
\renewcommand{\labelitemi}{$\bullet$}
\renewcommand{\labelitemii}{$\circ$}
\renewcommand{\labelitemiii}{$\cdot$}

%%
%% Set Path fuer Bilder
\graphicspath{{img/}}

%js definition-umgebung setzen!
\newtheorem{definition}{Definition}
% js end definition

%%
%%Quellcode Formatierungen Java
%%
\definecolor{javared}{rgb}{0.6,0,0} % for strings

\lstdefinestyle{javacode}{
	language=Java, 
   	basicstyle=\footnotesize, 
    	frame=lines,
   	keywordstyle=\bf{\color{blue!80!black!100}}, 
   	identifierstyle=, 
   	xleftmargin=15pt,
    	framexleftmargin=15pt,
    	tabsize=3,
   	frame=shadowbox,
    	rulesepcolor=\color{gray}, 
   	commentstyle=\color{green!50!black!100}, 
   	numberblanklines=false,
   	stringstyle=\color{javared}, 
   	breaklines=true, 
   	numbers=left, 
   	numberstyle=\tiny,
     numbersep=5pt,
    	breaklines=true,
    	showstringspaces=false,
}
%%
%%Quellcode Formatierung Android XML
\definecolor{darkgreen}{rgb}{0,0.5,0}
\lstdefinestyle{xmlcode}{
    language=xml,
    sensitive=true,
    tabsize=3,
    frame=lines,
    frame=shadowbox,
    rulesepcolor=\color{gray},
    xleftmargin=20pt,
    framexleftmargin=15pt,
    keywordstyle=\color{magenta}\bf,
    commentstyle=\color{darkgreen},
    stringstyle=\color{javared},
    numbers=left,
    numberstyle=\tiny,
    numbersep=5pt,
    breaklines=true,
    showstringspaces=false,
    basicstyle=\footnotesize,
    morecomment=[s]{<?}{?>},
    morecomment=[s]{<!-}{-->},
    alsoletter={<>=-}, 
    moredelim=[s][\color{black}]{>}{<},
    tag=[s],
    identifierstyle=\bf{\color{blue!80!black!100}},
    alsodigit={:},
}

%%
%%Quellcode Formatierung HTML
\lstdefinestyle{html5}{
    language=html,
    sensitive=true,
    tabsize=3,
    frame=lines,
    frame=shadowbox,
    rulesepcolor=\color{gray},
    xleftmargin=20pt,
    framexleftmargin=15pt,
    keywordstyle=\bf{\color{blue!80!black!100}},
    commentstyle=\color{darkgreen},
    stringstyle=\color{javared},
    numbers=left,
    numberstyle=\tiny,
    numbersep=5pt,
    breaklines=true,
    showstringspaces=false,
    basicstyle=\footnotesize,
    morecomment=[s]{<!-}{-->},
    alsoletter={<>=-}, 
    moredelim=[s][\color{black}]{>}{<},
    tag=[s],
    emphstyle={\color{magenta}},
    identifierstyle=\bf{\color{blue!80!black!100}}
}

