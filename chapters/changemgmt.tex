\chapter{Change Management}
\vspace{-0.4cm}
Verändern sich die Rahmenbedingungen eines Unternehmens, so muss sich das Unternehmen an diese dementsprechend anpassen. Wird beispielsweise eine neue gesetzliche Regelung eingeführt, die die Erhöhung des Mehrwertsteuersatzes vorschreibt, muss diese sowohl in die Geschäftsprozesse als auch in die komplette IT-Landschaft integriert werden. Damit diese Änderung durchgeführt werden kann, ohne das laufende IT-System negativ zu beeinflussen, muss dieser Änderungsprozess möglichst effizient gesteuert werden. Hierfür werden standardisierte Verfahren definiert, mithilfe dessen jede Veränderung hinsichtlich Notwendigkeit und Risiko umfassend analysiert wird. Ist eine Änderung erforderlich und kann diese ohne weitere Störungen durchgeführt werden, wird diese genehmigt. Ziel des Change Managements ist es also, alle Veränderungen, von der Registrierung bis hin zur Implementierung, zu kontrollieren, um das Risiko einer Instabilität möglichst minimal zu halten.
\\
Eine Änderung kann durch Störungen im laufenden Betrieb, bei der Benutzung oder auch durch Verbesserungsvorschläge, Erweiterungen, Kundenanfragen bzw. Rückfragen oder durch geänderte Anforderungen bzw. Geschäftsbedingungen hervorgerufen werden. Trifft eine Anfrage für eine Veränderung ein, wird der Change Management Prozess ausgelöst. 
Diese Veränderung betrifft in der Regel das Ändern, Hinzufügen oder das Entfernen von Komponenten des Konfigurationsmanagements wie beispielsweise Software, Hardware, Anwendungen oder Netzwerkkomponenten. 
Sowohl Kunden als auch Prozessbeteiligte und das Change Management selbst können eine Veränderung veranlassen. 
\vspace{-0.1cm}
\section{Der Change Management Prozess}
\vspace{-0.1cm}
Jeder Change muss den Prozess durchlaufen, unabhängig von der Schwere  seiner Auswirkung oder seiner Wichtigkeit. Dabei ist der Prozess so aufgebaut, dass ein Change immer für den nächstfolgenden Schritt freigegeben werden muss. 
Ziel dabei ist, alle Change-Aktivitäten zunächst umfassend auf potenzielle Gefahren zu untersuchen, bevor der Change in der Liveumgebung implementiert wird. So können unkontrollierte Störungen und Unterbrechungen in der IT-Umgebung vermieden werden.
Zudem sind alle Aktivitäten, die einen Change betreffen, zu dokumentieren. 
\\
Soll eine Veränderung an der IT-Infrastruktur umgesetzt werden, wird ein Request for Change durch den Antragsteller ausgelöst. Diese Anfrage wird anschließend an das Change Management weitergeleitet. Der Verantwortliche im Change Management Prozess prüft nun die Anfrage. Entspricht der Change nicht allen Qualitätsanforderungen oder lässt sich der Change nicht durchführen, wird der Request for Change abgelehnt. Ist der Change jedoch notwendig, um einen betrieblichen Ablauf gewährleisten zu können, wird der Request for Change registriert und für den nächsten Schritt freigegeben. 
\\
Bevor der Auftrag, eine Änderung an einem bestimmten Konfigurationselement vorzunehmen, genehmigt wird, muss dieser hinsichtlich seiner Dringlichkeit und Auswirkung auf die IT-Umgebung analysiert werden. Auf Basis dieser Analyse wird das weitere Vorgehen bestimmt.
\\
Hierbei wird dem Change zunächst eine Priorität zugewiesen. Die Priorität ergibt sich aus der Dringlichkeit der Durchführung des Changes und dem Ausmaß des Problems, welches der Change beheben soll. Muss die Durchführung eines Changes sofort erfolgen, hat dieser Change höchste Priorität. 
Dies ist dann der Fall, wenn ein Problem auftritt, dass die Funktionsweise des IT-Services enorm beeinträchtigt und somit schon einen gewissen Schaden verursacht hat. Es ist dementsprechend absolut notwendig, um einen ordnungsgemäßen Betrieb wiederherzustellen, dass der Change sofort in der Liveumgebung implementiert wird und nicht erst darauf gewartet wird, dass dieser nach einem langwierigen durchlaufen des Prozesses genehmigt wird. 
Hierbei müssen sofort entsprechende Maßnahmen eingeleitet werden. Dazu kann die Durchführung, weniger wichtiger Changes, zeitlich verzögert werden, um gewährleisten zu können, dass für die Durchführung dieses Changes genügend Ressourcen zur Verfügung stehen. Je nach Dringlichkeit, kann auf das Testen der Implementierung verzichtet werden. Changes mit einer niedrigen Priorität sind zwar wünschenswert aber nicht notwendig. Diese können gegebenenfalls verschoben und zu einem späteren Zeitpunkt durchgeführt werden.
\\
Anschließend wird dem Change eine Kategorie zugeteilt. Die jeweilige Kategorie gibt das Ausmaß an, inwieweit das laufende System durch den Veränderungsprozess belastet und gegebenenfalls in seiner Funktionalität beeinträchtigt wird. Je höher das Risiko der Anpassung eines Konfigurationselements an die geänderten Anforderungen ist, umso höher ist die zugeordnete Kategorie. Handelt es ich bei den Changes um Veränderungen, die regelmäßig durchgeführt werden, beispielsweise das Ändern eines Passworts, so wäre es unvorteilhaft, wenn diese Veränderung jedes Mal erneut genehmigt werden müsste. Solche Changes stellen für das Unternehmen kein großes Risiko dar. Aufgrund dessen werden solche Changes in die niedrigste Kategorie „Standard-Change“ eingeteilt. Alle Changes, die in die Kategorie Standard-Change fallen, müssen nicht mehr ausdrücklich vom Change Verantwortlichen genehmigt werden, sondern gelten schon als vorab autorisiert, um einen schnellen Durchlauf des Veränderungsprozesses garantieren zu können. Die Kosten des Standard-Changes können gut eingeschätzt werden. Zudem muss die Implementierung der Standard-Changes nicht immer von neu erfolgen. Der Change-Status im Change Record wird somit auf „Standard-Change“ gesetzt.
\\
Changes, die riskieren andere Komponenten des IT-Systems negativ zu beeinträchtigen, werden einer höheren Kategorie zugeordnet. Um jedoch beurteilen zu können, wie sich die Änderung eines Konfigurationselements auf die restliche IT-Infrastruktur auswirkt, muss zunächst festgestellt werden, in welcher Beziehung das jeweilige Konfigurationselement mit anderen Konfigurationselementen steht. An diesem Punkt kommt das Konfigurationsmanagement zum Einsatz. Das Konfigurationsmanagement liefert Informationen darüber,  welche Wechselwirkungen zwischen den einzelnen Konfigurationselementen bestehen. Beeinflusst der Change die Konfigurationselemente nur geringfügig, handelt es sich um einen Change mit geringem Risiko. Hier müssen noch keine Maßnahmen zur Absicherung des IT-Systems definiert werden. Zudem lässt sich die Veränderung ohne großen Aufwand umsetzen. Dieser Change fällt in die Kategorie 1. Alle Veränderungen dieser Kategorie werden vom Change Manager umfassend beurteilt und anschließend genehmigt. 

\begin{itemize}
\vspace{-0.1cm}
	\item \textit{Der Change Manager} leitet den kompletten Change Management Prozess. Es ist für einen ordnungsgemäßen Ablauf des Change Management Prozesses verantwortlich und überwacht alle Änderungsaktivitäten. Dabei dokumentiert er jeden einzelnen Change-Fortschritt. Er hat bei geringfügigen Changes die alleinige Befugnis diese für die nächste Phase freizugeben. Dazu führt er im Vorfeld für jeden Change eine umfassende Risikoanalyse durch und bewertet dessen Auswirkung auf den laufenden Betrieb. 
\end{itemize}
\vspace{-0.1cm}
Schwerwiegende Changes können nur mit großem Aufwand umgesetzt werden. Gerade deswegen, da im Problemfall mit Unterbrechungen im IT-Service gerechnet werden muss. Die Wahrscheinlichkeit, dass es zu Störungen im ordnungsmäßigen Ablauf der Geschäftsprozesse kommt, ist enorm. Hier müssen die Risiken ausreichend abgeschätzt werden und Maßnahmen entwickelt werden, um einen negativen Einfluss auf die IT-Services weitgehend zu vermeiden. Um sicherzustellen, dass dieser Change hinreichend beraten wird, wird dieser Change an das Change Advisitory Board (CAB) weitergeleitet. 
\begin{itemize}
	\item \textit{Das CAB} unterstützt den Change Manager bei risikoreichen Changes hinsichtlich der Entscheidung über die Annahme oder Ablehnung des Changes. Dazu setzt sich das CAB aus erfahrenen Personen zusammen, die sowohl die Business-Seite als auch den technischen Standpunkt vertreten. Mitglieder des CAB sind in der Regel Kunden, Anwender, IT-Experten, Entwickler und Tester, evtl. Subunternehmen oder Hersteller, sowie der Change Manager selbst. Die Zusammensetzung des CAB hängt dabei von dem zu diskutierenden Change ab. 
\end{itemize}
Je nachdem welche Kategorie und Priorität dem Change zugewiesen wurde, wird dieser durch die dafür zuständige Autorisierungsebene genehmigt. Dabei sind jedoch nicht nur die Auswirkung und Dringlichkeit von Bedeutung, sondern auch die Kosten und die Zeit der Umsetzung, als auch die zur Verfügung stehenden Kapazitäten und Ressourcen. 
\\
Wird ein Change genehmigt, beginnt die Planung des Changes durch den Change Manager. Dazu wird ein genauer Zeitplan erstellt, indem alle weiteren Aktivitäten klar definiert und zeitlich festgelegt sind. Dieser enthält unter anderem das vorgesehene Implementierungsdatum. Alle Aufgaben müssen klar verteilt sein. Zur optimalen Kontrolle, müssen alle durchgeführten Aktivitäten in dem Zeitplan dokumentiert werden.
\\
Anschließend wird der Change an die Entwicklungsabteilung weitergereicht, die für den Entwurf, die Implementierung und das Testen zuständig ist. Der Change Manager hat dabei nur noch eine Koordinationsfunktion, da nun andere Personen für die Umsetzung des Changes zuständig sind. Er überwacht alle Aktivitäten und stellt sicher, dass diese im vorgesehenen zeitlichen Rahmen ausgeführt werden.
\\
In der Entwurfsphase werden die Anforderungen an die durchzuführende Veränderung in einer technischen Lösung umgesetzt. Grundlegend ist die Strukturierung der Software. Dabei wird festgelegt, in welcher Beziehung die einzelnen Konfigurationskomponenten zueinander stehen und welches Verhalten diese haben. Der Entwurf dient als Vorlage, nach dem programmiert wird. Deswegen ist es essentiell wichtig, dass der Entwurf vollständig ist. Ist ein Entwurf unvollständig, kann es sein, dass in der Implementierung Optionen vergessen werden und später der Change nicht mehr den Anforderungen entspricht. 
\\
Die Implementierung liegt in den Händen von Technikern, die nun für die Durchführung des Changes verantwortlich sind. Das Change Management selbst arbeitet hier nur im Hintergrund und sorgt dafür, dass die Implementierung im dem festgelegten zeitlichen Rahmen erfolgt. Alle Implementierungsschritte müssen dokumentiert werden. 
\\
Die Implementierung liegt in den Händen von Technikern, die nun für die Durchführung des Changes verantwortlich sind. Das Change Management selbst arbeitet hier nur im Hintergrund und sorgt dafür, dass die Implementierung im dem festgelegten zeitlichen Rahmen erfolgt. Wurde die Implementierung durchgeführt, muss diese getestet werden. Inwieweit der Change getestet wird, hängt von der Kategorie ab. Getestet werden hierbei die Funktionen der Konfigurationskomponenten sowie der Programmcode. Zudem wird die gesamte Implementierungsphase genau analysiert. Wichtigstes Kriterium hierbei ist, ob mit dem Change die erwartete Veränderung an der IT-Infrastruktur erreicht wurde. Weitere Kriterien sind:
\begin{itemize}
	\item 	Traten bei der Implementierung und im gesamten Change-Prozess Störungen auf?
	\item 	Waren die zur Verfügung stehenden Ressourcen und das Budget ausreichend?
	\item Wie reagiert die IT-Infrastruktur auf den Change bzw. treten Qualitätsverluste auf?
\end{itemize}
Ist der Test nicht erfolgreich, muss das sogenannte Fall-Back Szenario eingeleitet werden. Dazu wurde im Vorfeld ein Fall-Back Plan erstellt. Dieser Plan erlaubt es, den vorherigen Zustand wiederherzustellen, falls die Implementierung fehlerhaft ist und nicht zum versprochenen Erfolg führt. Ist der Test erfolgreich, wird der Change durchgeführt und der Change-Management Prozess ist beendet.
\\
Zum Abschluss muss die Konfigurationsmanagementdatenbank hinsichtlich der Veränderung an den Konfigurationskomponenten aktualisiert werden, da möglicherweise neue Komponenten hinzugekommen sind, die in die Datenbank mitaufgenommen werden müssen. Das Konfigurationsmanagement muss immer über die aktuelle, gegebenenfalls geänderte, IT-Infrastruktur Bescheid wissen.
\\
Für ein Unternehmen ist es vorteilhaft, Methoden zu definieren, mithilfe dessen ein Change an einem oder mehreren Konfigurationskomponenten, kontrolliert und verwaltet werden kann. Dementsprechend können die Risiken leichter analysiert und eingeschränkt werden. Zugleich können Unterbrechungen und Qualitätsverluste im laufenden Betrieb gemindert werden. Geänderte Anforderungen können somit im Unternehmen schneller und effizienter zu geringen Kosten durchführt werden. 
\\
\cite{cm-oldenburg}, \cite{cm-stolpersteine}, \cite{cm-itil} \cite{cm-bestpractices}, \cite{cm-munich}