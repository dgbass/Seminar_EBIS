\chapter {Einleitung}

\section {Motivation}
Mit immer fortschreitender Globalisierung entwickelt sich auch die Struktur der Unternehmen weiter. Viele Unternehmen entwickeln sich hin zu multinationalen Konzernen. Hierbei müssen sich alle Bereiche den neuen Anforderungen anpassen. Dies betrifft auch die Softwareentwicklung innerhalb der Unternehmen. Um eine konzernweit einsetzbare Software zu entwickeln, setzten Unternehmen immer mehr auf eine Kooperation von Entwicklungsteams. Der gesamte Software-Entwicklungsprozess wird zerlegt und von verschiedenen Mitarbeitern oder Teams vollzogen. Was früher noch in einer Abteilung geschah, Passiert heute um den ganzen Globus verteilt. So ist es bei vielen Unternehmen Realität, dass ein Teil der Software aus Deutschland, ein anderer aus China und der dritte Teil aus Indien stammt. Hierbei kommt es schnell zu neuen Problemen, die durch die Zergliederung auftreten. Kommunikation wird als wichtige Voraussetzung zur Verwaltung der Ressource „Wissen“ angesehen. „Softwareentwicklung ist von Natur aus teambasiert“ \cite{einleitung}. In verteilten Projekten kommt der Kommunikation eine noch größere Bedeutung zu. Es muss Wissensaustausch über Landes- oder sogar Kontinentgrenzen hinweg betrieben werden, was oft eine schwere Aufgabe darstellt. So hat \acs{z. B.} eine Studie aufgezeigt, dass persönliche Treffen nicht durch Kommunikationsmedien ersetzt werden können. Trotz Videokonferenzen und  Life-Chats, ist die Face-to-Face Kommunikation durch Reisen unabdingbar. Durch die verteilte Arbeit mehrerer Personen, rückt aber auch die Verwaltung der Versionen von Programmen und Dokumenten in den Vordergrund. Es müssen Regeln definiert werden, wer etwas ändern kann, wann dies geschehen darf oder in welcher Reihenfolge. 
\section{Gliederung}
Diese Arbeit greift zuerst das Thema Software Configuration Management auf und behandelt seine einzelnen Aspekte.
Zuerst wird die Konfigurationsidentifikation behandelt, anschließend folgt das Build Management, das Versionsmanagement, ebenso wie das Release und schließlich das Change Management.
