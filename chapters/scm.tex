\chapter{Software Configuration Management (\acs{SCM})}
Software Configuration Management bedeutet, die Evolution einer Software zu jeder Zeitpunkt des Produktlebenszyklus zu begleiten und zu kontrollieren. \acs{SCM} als Softwareentwicklungsdisziplin, die die Werkzeuge und Techniken bzw. Methoden beinhaltet, die zum Erzeugen, Verwalten, Erweitern und Warten von Softwarebestandteilen benutzt wird, ist eine etwas formalere Beschreibung. Die Notwendigkeit der Kontrolle des Prozesses der Softwareentwicklung wird von den meisten Unternehmen erkannt, doch scheitert es meist an der optimalen Umsetzung. So findet das Thema SCM in der Literatur und auch der Forschung eine eher geringe Aufmerksamkeit und sind die zu lösenden Probleme den meisten jungen Akademikern nicht präsent und finden deshalb auch auf dem Markt wenig Nachfrage. SCM zeichnet gute Softwareentwicklung aus.  Es verbessert die Qualität und Ausfallsicherheit von Software durch Strukturen zur Identifizierung und Kontrolle von Dokumentation, Code, Interfaces, und Datenspeicher. Auch die Unterstützung von Methoden, angepasst an die Anforderungen, Standards, Organisationsstrukturen und Management-Philosophie ist wichtig. Ein anderer Punkt, der zur Verbesserung beiträgt, sind Statusmeldungen über Baseline, Change Control, Test, Release, sowie Auditing, die für Produktmanagement und zur Information über das Produkt wichtig sind. 

\section{Herkunft von SCM}
Software ist sehr schnell und einfach zu zu erstellen und zu ändern. Aber es ist von großer Bedeutung, nachvollziehen zu können, welche Version was beinhaltet und welche Komponenten zusammen ein funktionsfähiges System ergeben. Dies wird besonders dann ersichtlich, wenn man das Konfigurationsmanagement (KM) statt dem SCM betrachtet. Das SCM ist von diesem, in den 70er Jahren, abgeleitet und auf Software übertragen worden. KM wird von der ISO folgendermaßen definiert: „Konfigurationsmanagement ist eine Managementtätigkeit, die die technische und administrative Leitung des gesamten Produktlebenszyklus, der Konfigurationseinheiten des Produktes und der produktkonfigurationsbezogenen Angaben übernimmt“ (\acs{ISO} 10007: 2003 \cite{km-hamburg}). Der amerikanische Industriestandard EIA-649 definiert KM etwas verständlicher: „Verfahren zur Herbeiführung und ständiger Sicherstellung der Übereinstimmung der Leistungs-, Funktions- und physischen Charakteristiken eines Produkts mit den zugehörigen Anforderungen, den Ausführungen, den Ausführungsunterlagen und den für den Betrieb erforderlichen Informationen während des gesamten Lebenszyklus des Produkts.“ \cite{gbt-km}

\section{Vergleich mit KM}
Nimmt man, statt dem abstrakten Produkt Software, die Hardware als Beispiel wird dies gleich viel klarer. So kann jeder leicht nachvollziehen, dass nicht jedes Teil mit jedem in einem Computer beliebig kombinierbar ist, sondern bestimmte Konfigurationen andere ausschließen. Man stelle sich Bestandteile eines Computers vor, z.B.: Mainboard, Prozessor, einen Speicher, Maus, Monitor, usw. Hat man jetzt einen Computer, ohne VGA-Anschluss, so kann man den Monitor, der nur einen solchen Anschluss besitzt, nicht anschließen und folglich auch nicht verwenden, da die beiden Systeme nicht kompatibel sind. Hieraus entsteht also kein funktionsfähiges System. So gibt es bei einem Computer viele einzelne Bestandteile, die in der passenden Konfiguration vorliegen müssen, um am Ende ein arbeitsfähiges System zu ergeben. Dieser Sachverhalt lässt sich auch auf Software übertragen. So muss auch zwischen den verschiedenen Teile, passende Verbindungen, sog. Interfaces, geschaffen werden, sodass diese zusammenführbar werden. So muss sich auch das SCM mit denselben Problemen wie das KM der Hardware beschäftigen und noch vielen mehr. Ein Softwarebestandteil hat ein Interface, Diese Softwareteile werden auch Subsystem, Modul oder Komponente genannt und haben jeweils ein Interface. Viele Subsysteme zusammen und ergeben ein Softwaresystem. Jedes dieser Subsysteme muss eindeutig identifizierbar sein und eine Versionsnummer aufweisen. Schlussendlich benötigt man eine Stückliste über die Versionen aller Bestandteile, das das Gesamtsystem bilden. Allerding ist es viel schwerer, SCM richtig zu machen, da hier die physikalischen Grenzen fehlen, die es beim KM gibt. Zudem ist Software viel einfacher und schneller änderbar. Schon ein paar Klicks und eine neue Version ist geschaffen. Im  Gegensatz zur Hardwareproduktion, ist die Softwareentwicklung sehr viel schneller und kann an einem Tag von mehreren Teammitgliedern 100fach  betrieben werden. So entsteht schnell eine riesige Flut an Versionen, die es zu verwalten gilt. Am Ende des Prozesses gilt es, alle Teile am selben Ort zur selben Zeit zusammenzuführen, sodass ein System entsteht, dass wie gefordert funktioniert. Die Bedeutung von SCM wird meist erkannt und verstanden, doch scheitert es meist an der Umsetzung. SCM ist sehr komplex und es gibt keine genauen Richtlinien und Grundsätze für ein gutes SCM-System. 

\section{Aufgaben des SCM}
Die Aufgaben des \acs{SCM} sind vielseitig und bestehen aus der Verwaltung aller Komponenten eines Softwaresystems meist in einem Repository. Es soll keine redundanten Kopien oder Versionen geben und diese sicher gespeichert sein. Die Unterschiede zwischen den Versionen soll hier sichtbar gemacht werden. Es soll damit Änderungen leichter ersichtlich machen, sodass diese diskutiert und dann verworfen oder angenommen werden können. Und es werden zusätzlich zu jeder Änderung Metadaten gespeichert. Diese beinhalten Daten über den Autor der Änderung, den Zeitpunkt, warum etwas geändert wurde und wo. Diese Daten sind wichtig, um die Versionen verständlich und auch später noch nachvollziehbar zu machen. Es macht durch die persistente Speicherung auch den Zugriff auf Vorgängerversionen gut zugänglich. Und erlaubt auch die Definition von Referenzversionen, \acs{z. B.} Releases, Milestones oder wichtige Zwischenstände wie bspw. fehlerfrei getestete Versionen.
\\
Es gehören eine ganze Menge Aufgaben dazu, wie das Konfigurationsidentifikation, Build, Release und das Change Management, welche in den folgenden Kapiteln näher erläutert werden.
\cite{scm-unibonn}
\section{Relevanz}
Es ist sehr wichtig für ein gutes \acs{SCM} die individuellen Anforderungen zu identifizieren und das Konzept an diesen anzupassen. Die Zielvorstellung, sowie das erwähnte Konzept, dürfen nicht vernachlässigt werden und sollten immer präsent sein, um geeignete Werkzeuge unterstützend einsetzten zu können. Diese sollen vor allem die Effektivität und die Effizienz des gesamten Softwareentwicklungsprozess steigern und somit zu einer höheren Kundenzufriedenheit führen.
\\\\
\cite{scm-agil}, \cite{scm-pearson}