%%
%% hier Namen etc. einsetzen
%%
\newcommand{\nameone}{Sara Steisslinger}
\newcommand{\matnrone}{Matrikelnummer: 823825}
\newcommand{\stdrichtungone}{WiWi BSc}
\newcommand{\stdrichtungtwo}{WiWi MSc}

\newcommand{\nametwo}{Jasmin Klose}
\newcommand{\matnrtwo}{Matrikelnummer: 757415}
\newcommand{\namethree}{Manuel Buchert}
\newcommand{\matnrthree}{Matrikelnummer: 758373}
\newcommand{\namefour}{Florian Rotter}
\newcommand{\matnrfour}{Matrikelnummer: 761110}
\newcommand{\namefive}{Daniel Glunz}
\newcommand{\matnrfive}{Matrikelnummer: 702033}
\newcommand{\namesix}{Manuel Ott}
\newcommand{\matnrsix}{Matrikelnummer: 770969}


\newcommand{\titel}{Dokumenten- /Konfigurationsmanagement in verteilten Software-Projekten}
\newcommand{\semester}{Sommersemester 2014}
\newcommand{\gutachter}{Prof.\ Dr.\ Franz Schweiggert}
\newcommand{\fakultaet}{Fakultät für Mathematik und Wirtschaftswissenschaften}
\newcommand{\institut}{Institut für angewandte Informationsverarbeitung}
\newcommand{\arbeit}{Begleit-Seminar zur Vorlesung\\Entwicklung und Betrieb von Informationssystemen}

%%
%% Setzt Autor und Titel in den Metadaten des erzeugten Dokumentes
%%
\pdfinfo{
    /Author (\nameone \nametwo \namethree \namefour \namefive)
    /Title (\titel)
    /Producer (pdfeTex 3.14159-1.30.6-2.2)
    /Keywords ()
}
\hypersetup{
    pdftitle=\titel,
    pdfauthor=\nameone \nametwo \namethree \namefour \namefive
    pdfsubject={\arbeit},
    pdfproducer={pdfeTex 3.14159-1.30.6-2.2},
    colorlinks=false,
    pdfborder=0 0 0
}
